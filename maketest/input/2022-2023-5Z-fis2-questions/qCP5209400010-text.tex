{\bf Text @TEXT@. Electric Current}

Electric current is a flow of electric charge carried by moving 
electrons in a wire. The electric current is created by
electrons or charges continuously moving through a path called 
an electric circuit. 
It flows from a power source like a battery or power station.

A closed circuit has a complete path for current to flow allowing 
the electric charges or electrons to flow through the
wires of the circuit. An open circuit will not allow the electric 
charges or electrons to flow through the wires of the
circuit. A switch can be used to open and close a circuit.

In a series circuit, the same current flows through each of the 
components. In a series circuit, each bulb will receive
the same electrical charge, but if one goes out, all will go out. An 
example of a series circuit may be a string of
Christmas lights. If any one of the bulbs is missing or burned out, 
no current will flow and none of the lights will go on.

Batteries are also a source of electric current usually used with a 
series circuit. The electric current from the
battery flows in one direction to the component such as a radio, 
flashlight, or a toy.

Parallel circuits will have different amounts of current flowing 
through them. The same voltage is applied to parallel
circuits, but different amounts of current will flow through the 
wires. Voltage is a kind of electrical force that
makes electricity move through a wire and it is measured in 
volts. The higher the voltage, the more current will
tend to flow. A 12-volt car battery will normally produce more 
electric current than a 1.5-volt flashlight battery.

A parallel circuit example is the wiring of a house. There is a 
single power source supplying all the lights and
appliances with the same voltage. However, if one of the lights 
burns out, the current will still flow through the rest
of the house.

There are power plants that produce electricity for homes and 
businesses. Most power plants use coal to generate
electricity, but some use wind, water, or natural gas. The power 
grid is the system connecting all of the power plants
across the country. All the poles and wires along the highway and 
roads are a part of the power grid. A transformer can
help in decreasing or increasing the voltage as the electricity 
travels to homes and businesses through transmissions
lines. A meter is used to measure the amount of electricity used.

The electricity goes through wires to the service panel in a 
basement or garage, where breakers or fuses protect the
wires inside a house from being overloaded. The electricity then 
travels through wires inside the walls to outlets and
switches all over the house.

Conductors are made of materials that electricity can flow 
through easily. A material that is a good conductor gives
very little resistance to the flow of electricity. The electricity can 
flow through a conductor very easily. Examples
of conductors include water, trees, aluminum, copper, people, 
and animals.

Insulators prevent or block the flow of electricity. Insulators do not 
allow the flow of electricity and block the
electricity from moving along its path. Examples of insulators are 
glass, rubber, porcelain, and plastic. Wires that
carry electricity are covered with an insulator.

There are many steps involved when electric current flows from 
its source to its use.

