Nella figura @FIGURE@ alla fine del testo,
si determini dove si deve posizionare il punto $P$, 
sul segmento $AB$ di lunghezza 13 cm, in modo che 
il poligono $APEF$ abbia la stessa area del poligono $PEDCB$.
Si ponga $AP=x$. 
Suggerimento: l'area di $APEF$ deve essere la met\`a 
dell'area di $ABCDEF$; quest'ultima si pu\`o calcolare 
come differenza tra l'area del trapezio rettangolo $ABGF$, 
dove $G$ \`e l'incrocio del prolungamento di $FE$ con il 
prolungamento di $BC$, e l'area del rettangolo di lati 
$ED$ e $DC$. L'area di un trapezio \`e $\frac{(B + b)\cdot h}{2}$.