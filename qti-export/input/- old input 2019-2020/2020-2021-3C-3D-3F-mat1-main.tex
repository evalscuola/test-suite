\documentclass[10pt,a4paper]{article} 
\pagestyle{empty}

\usepackage{tikz}
\newcommand*\circled[1]{\tikz[baseline=(char.base)]{
%            \node[shape=circle,draw,inner sep=2pt] (char) {#1};}}
            \node[shape=circle,draw,inner sep=2pt] (char) {$\phantom{8}$};
            \node[draw=none,fill=none] (char) {#1};}}
            
\newcommand*\squared[1]{\tikz[baseline=(char.base)]{
            \node[shape=rectangle,draw,inner sep=3pt] (char) {$\phantom{A}$};
            \node[draw=none,fill=none] (char) {#1};}}


            
\usepackage{enumitem}



\usepackage{amsmath}
\usepackage{amssymb,amsfonts,textcomp}
% \usepackage{amsfonts}
 \usepackage{latexsym}
 \usepackage{epsfig}
 \usepackage{graphicx}
 \usepackage{fancyhdr}
 \usepackage{verbatim}
 
 \usepackage{multicol}
% \usepackage{flushend}
% \usepackage{lipsum}

 \usepackage{cancel}
 
 \fancyhf{}
 \renewcommand\headrule{}
 
%\usepackage[latin1]{inputenc}
\usepackage[utf8]{inputenc}

%%%%% TIMES
\usepackage{txfonts}
\usepackage{mathptmx}

\everymath{\displaystyle}


%%%%% MATHPAZO
%\usepackage[sc]{mathpazo}


%%%%%  LIBERTINE
%\usepackage{libertine}
%\usepackage[libertine,cmintegrals,cmbraces,vvarbb]{newtxmath}
%\usepackage[scaled=0.95]{inconsolata}



\usepackage{eurosym}


\usepackage[italian]{babel}



\addtolength{\textwidth}{+66mm}
\addtolength{\oddsidemargin}{-34mm}
\addtolength{\textheight}{+64mm}
\addtolength{\topmargin}{-34mm}

\newcommand{\mybox}{\fbox{$\phantom{\frac{M}{M}}\qquad\qquad\qquad \qquad$}$\ $}
\newcommand{\longmybox}{\fbox{$\phantom{\frac{M}{M}}\ \ \ \qquad \qquad \qquad\qquad\qquad \qquad \qquad$}$\ $}
\newcommand{\shortmybox}{\fbox{$\phantom{\frac{M}{M}}\qquad\qquad\quad $}$\ $}
\newcommand{\veryshortmybox}{\fbox{$\phantom{\frac{M}{M}}\qquad\qquad $}$\ $}


\title{\textsc{Liceo Statale ``Niccolò Machiavelli'' di Roma}\\
			3$^\circ$ anno -- Verifica di Matematica n. 1
			-- Divisione e Fattorizzazione dei Polinomi}
\author{}
\date{} 

\pagestyle{fancy}

\newcommand{\cotan}{\mathrm{cotan\,}}%\,}
%\newcommand{\ee}{\mathrm{e}}
\newcommand{\Log}{\mathrm{Log}\,}




\begin{document}
\thispagestyle{empty}




\maketitle
\vspace{-1cm}

%\noindent
%{\bf %@date@ -- Classe @class@ 
%Versione @ver@}

%\begin{comment}
\noindent
\begin{tabular}{p{2.1cm} r r r }
{\bf %@date@ -- Classe @class@ 
Versione @ver@} & Data \shortmybox $\ $& Classe \shortmybox $\ $ & Alunno \longmybox  \\
\end{tabular}
%\end{comment}

\vspace{1cm}


\thispagestyle{fancy}

\noindent 
Il punteggio 
per ogni risposta corretta è di 4 punti,
%riportato all'inizio del quesito.
%Inoltre saranno assegnati %4 punti per ogni risposta corretta, 
per ogni risposta omessa 1 punto, per ogni risposta sbagliata 0 punti. % e 
La durata della verifica è di cinquanta minuti.
%La durata della verifica è di un'ora.
{\bf In ogni caso saranno valutate esclusivamente le risposte riportate nella tabella.}  %in allegato
% solo
%a seguire.}
%\vspace{.5cm}

\tikz[remember picture,overlay] \node[inner sep=0pt, shift={(-4 cm,-1cm)}] at (current page.north east){\includegraphics[width=5.6cm]{buon-compito.pdf}};%{\Large $\mathscr{Buon compito}$};



\vspace{0cm}
\noindent
\begin{center}
%\vspace{-1cm}
\begin{tabular}{ | c | c | c | c | c | c | c | c | c | c | c | c |%
 c | c | c | c | c | c | c | c |}
\hline
\multicolumn{20}{|l|}{ {\bf Risposte}} \\ %Tabella delle risposte
\hline
1 &  2 &  3 &  4 &  5 & 
6 &  7 &  8 &  9 &  10 & 
 11 &  12 &  13 &  14 &  15 & 
 16 &  17 &  18 &  19 &  20 
\\
\hline
$\ $ & $\ $ & $\ $ & $\ $ & $\ $ & 
$\ $ & $\ $ & $\ $ & $\ $ & $\ $ & 
$\ $ & $\ $ & $\ $ & $\ $ & $\ $ &
$\ $ & $\ $ & $\ $ & $\ $ & $\ $
\\
\hline
21 & 22 & 23 & 24 & 25 & 26 & 27 & 28 & 29 & 30 & 
31 & 32 & 33 & 34 & 35 & 36 & 37 & 38 & 39 & 40
\\
\hline
$\ $ & $\ $ & $\ $ & $\ $ & $\ $ & 
$\ $ & $\ $ & $\ $ & $\ $ & $\ $ & 
$\ $ & $\ $ & $\ $ & $\ $ & $\ $ &
$\ $ & $\ $ & $\ $ & $\ $ & $\ $
\\
\hline
\end{tabular}
\end{center}
\vspace{.2cm}



\begin{multicols}{2}

\begin{enumerate}[label=\protect\circled{\arabic*}]

@QUESTIONS@

\end{enumerate}

\end{multicols}




\vspace{.5cm}
\begin{multicols}{3}
	\noindent

@FIGURES@

%
\end{multicols}



\vspace{.5cm}

@TABLES@


\vfill

{\bf %@date@ -- Classe @class@ 
Versione @ver@}
\begin{comment}
\begin{tabular}{p{2.1cm} r r r }
{\bf %@date@ -- Classe @class@ 
Versione @ver@} & Data \shortmybox $\ $& Classe \shortmybox $\ $ & Alunno \longmybox  \\
\end{tabular}
\end{comment}



\end{document}
%

