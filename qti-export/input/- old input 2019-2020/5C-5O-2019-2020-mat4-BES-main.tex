\documentclass[11pt,a4paper]{article} %a4paper,
\pagestyle{empty}

\usepackage{tikz}
\newcommand*\circled[1]{\tikz[baseline=(char.base)]{
%            \node[shape=circle,draw,inner sep=2pt] (char) {#1};}}
            \node[shape=circle,draw,inner sep=2pt] (char) {$\phantom{8}$};
            \node[draw=none,fill=none] (char) {#1};}}
            
\newcommand*\squared[1]{\tikz[baseline=(char.base)]{
            \node[shape=rectangle,draw,inner sep=3pt] (char) {$\phantom{A}$};
            \node[draw=none,fill=none] (char) {#1};}}

%\def\checkmark{\tikz\fill[scale=0.4](0,.35) -- (.25,0) -- (1,.7) -- (.25,.15) -- cycle;}

            
\usepackage{enumitem}



 \usepackage{amsfonts}
 \usepackage{latexsym}
 \usepackage{epsfig}
 \usepackage{graphicx}
 \usepackage{fancyhdr}
 \usepackage{verbatim}
 
 \usepackage{multicol}
 \usepackage{cancel}
 \usepackage{bm}
 
 
 \fancyhf{}
 \renewcommand\headrule{}
 
%\usepackage[latin1]{inputenc}
\usepackage[utf8]{inputenc}

\usepackage{txfonts}

\usepackage{mathptmx}

\usepackage{eurosym}


\usepackage[italian]{babel}



\addtolength{\textwidth}{+30mm}
\addtolength{\oddsidemargin}{-16mm}
\addtolength{\textheight}{+44mm}
\addtolength{\topmargin}{-24mm}

\newcommand{\mybox}{\fbox{$\phantom{\frac{M}{M}}\qquad\qquad\qquad \qquad$}$\ $}
\newcommand{\longmybox}{\fbox{$\phantom{\frac{M}{M}}\ \ \ \qquad \qquad \qquad\qquad\qquad \qquad \qquad$}$\ $}
\newcommand{\shortmybox}{\fbox{$\phantom{\frac{M}{M}}\qquad\qquad\quad $}$\ $}
\newcommand{\veryshortmybox}{\fbox{$\phantom{\frac{M}{M}}\qquad\qquad $}$\ $}


\title{\textsc{Liceo Statale ``Niccolò Machiavelli'' di Roma}\\
			5$^\circ$ anno -- Verifica di Matematica n. 4 -- Le derivate}
\author{}
\date{} 

\pagestyle{fancy}

\newcommand{\cotan}{\mathrm{cotan\,}}%\,}
%\newcommand{\ee}{\mathrm{e}}


\newcommand{\Log}{\mathrm{Log}\,}



\begin{document}
\thispagestyle{empty}
\raggedright


%\vspace{-2cm} no effect
\maketitle
\vspace{-1cm}

\noindent
\begin{tabular}{l}
\sc \large Versione $\heartsuit$@ver@ \\
\\
\sc \large Data \shortmybox $\ $ $\ $ \sc \large Classe \shortmybox $\ $ \\
\\
\sc \large Alunno \longmybox \\
\end{tabular}


\vspace{1cm}


\sc
\large

\thispagestyle{fancy}
%$\textwidth$
%\circled{10}


Il punteggio 
%per ogni risposta corretta 
è di 4 punti per ogni risposta corretta,
%riportato all'inizio del quesito.
%Inoltre saranno assegnati %4 punti per ogni risposta corretta, 
1 punto per ogni risposta omessa e 0 punti per ogni risposta sbagliata.
La durata della verifica è di un'ora.
{\bf \sc Si suggerisce di ricopiare le risposte nella tabella
a seguire.
}

\tikz[remember picture,overlay] \node[inner sep=0pt, shift={(-4 cm,-1cm)}] at (current page.north east){\includegraphics[width=5.6cm]{buon-compito.pdf}};

\vspace{.5cm}
\noindent
\begin{center}
%\vspace{-1cm}
\begin{tabular}{ | c | c | c | c | c | c | c | c | c | c | }
\hline
\multicolumn{10}{|l|}{ {\sc Tabella delle risposte}} \\
\hline
1 &  2 &  3 &  4 &  5 & 
6 &  7 &  8 &  9 &  10 \\
\hline
$\quad\ \quad $ & $\quad\ \quad $ & $\quad\ \quad $ & $\quad\ \quad $ & $\quad\ \quad $ & 
$\quad\ \quad $ & $\quad\ \quad $ & $\quad\ \quad $ & $\quad\ \quad $ & $\quad\ \quad $ 
\\
\hline
11 &  12  & 13 &  14 &  15 & 16 &  17 &  18 &  19 &  20  
\\
\hline
$\quad $ & $\quad $ & $\quad $ & $\quad $ & $\quad $ & 
$\quad $ & $\quad $ & $\quad $ & $\quad $ & $\quad $ 
\\
\hline
21 &  22 &  23 &  24 &  25 & 
26 &  27 &  28 &  29 &  30 \\
\hline
$\quad\ \quad $ & $\quad\ \quad $ & $\quad\ \quad $ & $\quad\ \quad $ & $\quad\ \quad $ & 
$\quad\ \quad $ & $\quad\ \quad $ & $\quad\ \quad $ & $\quad\ \quad $ & $\quad\ \quad $ 
\\
\hline
31 &  32  & 33 &  34 &  35 & 36 &  37 &  38 &  39 &  40  
\\
\hline
$\quad $ & $\quad $ & $\quad $ & $\quad $ & $\quad $ & 
$\quad $ & $\quad $ & $\quad $ & $\quad $ & $\quad $ 
\\
\hline
\end{tabular}
\end{center}
\vspace{.5cm}


%\begin{multicols}{2}

\begin{enumerate}[label=\protect\circled{\arabic*}]



@QUESTIONS@


\end{enumerate}


\vspace{.5cm}
%\begin{multicols}{4}
	\noindent

@FIGURES@

%
%\end{multicols}





\begin{comment}

\vfill

\begin{tabular}{p{9cm} r }
{\bf 23 maggio 2017 -- Classi prime -- Versione $\heartsuit$@ver@} &  
{Alunno } \longmybox  \\
\end{tabular}

\end{comment}



\end{document}
%

